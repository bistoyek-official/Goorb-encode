\documentclass[oneside]{book}
\usepackage[utf8]{inputenc}
\usepackage{hyperref}
\usepackage{mathtools}
\usepackage{listings}
\usepackage{xcolor}  
\usepackage{graphicx}
\usepackage{titlesec}
\usepackage{cleveref}
\usepackage{fancyhdr}
\usepackage{array}
\usepackage{colortbl}
\usepackage{booktabs}

\pagestyle{fancy}
\fancyhf{}
\fancyhead[L]{GOORB ENCODE}
\fancyhead[R]{\thepage}
\renewcommand{\headrulewidth}{0pt}

\hypersetup{
    colorlinks=true,
    linkcolor=black,
    filecolor=magenta,
    urlcolor=cyan,
    citecolor=green
}

\titleformat{\chapter}[hang]{\Huge\bfseries}{}{0pt}{}
\titleformat{\section}[hang]{\Large\bfseries}{}{0pt}{\thesection \space}

\renewcommand{\thesection}{\arabic{section}}

\lstset{
    language=C++,
    backgroundcolor=\color{black!5},
    basicstyle=\ttfamily\footnotesize,
    keywordstyle=\color{blue}\bfseries,
    commentstyle=\color{green},
    stringstyle=\color{red},
    numbers=left,
    numberstyle=\tiny\color{gray},
    stepnumber=1,
    numbersep=10pt,
    showspaces=false,
    showstringspaces=false,
    showtabs=false,
    frame=single,
    tabsize=4,
    captionpos=b,
    breaklines=true,
    breakatwhitespace=false,
    title=\lstname
}

\newcommand{\myparagraph}[1]{\paragraph{\textnormal{#1}}}

\setcounter{section}{-1}

\begin{document}

\title{Goorb encode\\(A bistoyek R.I.C. Research Project)}
\author{Kasra Fouladi}
\date{12/14/2024}

\maketitle

\frontmatter
\tableofcontents
\mainmatter

\small

\section{Introduction}

\subsection{Acknowledgments}

\myparagraph{
This article was written entirely by me without any external assistance. I am grateful for the attention and support of all my friends and colleagues.
}

\subsection{Importance}

\myparagraph{
In today’s world, we face an increase in data exchange on digital platforms and the rapid advancement of attack methods, supercomputers, and quantum attacks. There is a significant risk that a particular cipher-text could be broken, causing massive damages in the future. The need for strong and efficient encryption and hash algorithms is more critical than ever.
}
\footnote{cipher-text: An encoded text}

\myparagraph{
This urgency drives cryptography specialists to make considerable efforts to create more secure encryption and hash algorithms. These advancements aim to ensure that even if hackers gain unauthorized access to information, they cannot easily retrieve the original data. In this article, I introduce a new encryption method that uses mathematical and algorithmic techniques to resist various attacks, especially quantum attacks, thereby keeping information secure for future decades.
}

\subsection{Abstract}

\myparagraph{
As previously mentioned, one of the primary goals of creating this encryption method is to ensure security against numerous attacks. Therefore, I have invented a new encryption style called \textbf{Goorbian encryption}, which utilizes one-way functions, an algorithm as a private key, and a cipher-text bank as a public key. This approach offers significant advantages in terms of security, and by employing this style, any encryption algorithm will be resistant to many attacks. In the remainder of the article, I will address each of these topics in detail.
}
\footnote{For more information about public and private keys:\\ \href{https://ssldragon.com/blog/public-key-cryptography}{https://ssldragon.com/blog/public-key-cryptography}}

\myparagraph{
This encryption method offers significant flexibility, meaning you can use any one-way function depending on your specific application. In this research, I introduced a messaging protocol using this method named \textbf{Raz} and \textbf{Goorb post} a messenger using this protocol, two versions of this style, \textbf{Goorb encode 2} using the SHA-256 algorithm and \textbf{Goorb encode 1} using an algorithm similar to Bubble Shooter in its structure of one-way function of my encryption.
}
\footnote{For more information about SHA-256:\\ \href{https://networkencyclopedia.com/sha-256-unmasked-deciphering-cryptographic-hash-functions}{https://networkencyclopedia.com/sha-256-unmasked-deciphering-cryptographic-hash-functions}}
\footnote{For more information about Bubble Shooter game:\\ \href{https://games.skillz.com/guides/bubble-shooter}{https://games.skillz.com/guides/bubble-shooter}}

\section{Goorbian encryption}

\myparagraph{
To better understand the \textbf{Goorbian encryption} mechanism, it is first necessary to know about this method's two main parts: the $F$ function and the cipher-text bank.
}
\footnote{Goorb is a word originated from gorbeh, which means cat in persian.}

\subsection{Function}

\myparagraph{
This method consists of two main parts: the function (let's call it $F$) and the cipher-text bank. First, I explain the function part. Imagine a function $F$ from $A$ to $B$, where $A$ is an infinite set and $B$ is finite, and $F$ distributed $A$ almost uniformly into $B$, it means for any $b \in B$ there are an infinite number of $a \in A$ which $F(a) = b$. For instance, $A$ could be the set of real numbers and $B$ could be the set of alphabet letters. This function is clearly one-way. If $A$ is the set of cipher-texts and $B$ is the set of plain-texts, you can encode the same plain-text into an infinite number of cipher-texts. This makes this method immune to many attacks such as crypt-analysis attacks, statistical-analysis attacks, especially machine learning attacks, and even quantum attacks.
}
\footnote{In section 4 I examined all of these attacks and show why they are ineffective.}

\myparagraph{
Here is a very simple C/C++ pseudo-code to better understand the F function:
}

\,

\begin{lstlisting}
//An example of cipher-blocks (elements of A):
struct cipher_block{
    vector<int> v;
};
//Private key is a part (probably all) of F structure
int key(int num){
    return num % 256;
}
//A simple example of F function (from A to B):
int F(cipher_block a){
    int res = 0;
    for(int i = 0; i < a.v.size(); ++i)
        res += a.v[i];
    int b = key(res) //b belongs to B = {0, 1, ..., 255}
    return b;
}
\end{lstlisting}

\myparagraph{
Decoding the encoded objects is straightforward by calculating $F(a)$, where $a$ is a block in the encoded file. However, encoding something is a different journey. Due to the complexity of the $F$ mechanism, you probably cannot directly find an element $a$ in $A$ such that $F(a) = b$ for a particular element $b$ in $B$. Instead, you need to create a bank of cipher-texts and split them into equivalence categories. In each category, every two cipher-block have the same $F$ result, and no two cipher-block in different categories have the same $F$ result.
}

\subsection{Bank}

\myparagraph{
As mentioned earlier, due to the complexity of $F$, it is highly improbable to find an element from $A$ whose $F$ result equals a particular element from $B$ on the first attempt. Given that for any element $b$ in $B$, there are infinitely many elements $a$ in $A$ such that $F(a) = b$, the probability of $F(a)$ equaling any specific element in $B$ is uniformly distributed. Therefore, by testing a large number of elements from $B$, we can construct a cipher-text bank where for each element in $B$, there is at least one corresponding element in $A$ such that $F(a) = b$. According to the \textbf{Goorbian Probability Distribution}, if $n$ elements are randomly tested from $A$, it is likely that the bank will be complete ($n$ is the answer to the following equation).
}
\footnote{Goorbian Probability Distribution is explained in section 3.}

$$\frac{(|B| - 1) ^ n}{|B| ^ {n - 1}} = 0.1$$

\myparagraph{
Since it's impossible for computers to have infinite memory and it's better to minimize $\frac{|\text{cipher-text}|}{|\text{plain-text}|}$ , it's possible to choose cipher-blocks from an extremely huge finite subset of $A$ provided that they are mapped almost uniformly into elements of $B$ by $F$.
}

\myparagraph{
Here is a C/C++ simple pseudo-code that following previous one to better understand the cipher-block bank concept and decode and encode mechanism in Goorbian encryption:
}

\,

\begin{lstlisting}
//This code follows the previous one
const int B = 256; //It's the size of B

vector<cipher_block> bank[B];
//Equivalence categories are bank[i] (0 <= i < B)
//For any 2 cipher-blocks in bank[b] like a, F(a) = b

cipher_block random_block(){
    cipher_block res;
    int sz = 10 + rand() % 10; //sz is size of the vector res.v
    //Now there are 2^(32*10) + ---- + 2^(32*19) possible outputs
    for(int i = 0; i < sz; ++i)
        res.v.push_back(rand()); //A random 32-bit integer
    return res;
}
//At first you have to create a bank and it's your public key
void new_bank(){
    for(int i = 0; i < B; ++i)
        bank[i].clear();
    bool mark[B] = {};
    //If mark[i] is false then bank[i] is empty 
    int zero = B;
    //Number B's elements like b that bank[b] is empty
    srand(time(nullptr));
    while(zero){
        cipher_block a = random_block();
        int b = F(a);
        if(!mark[b])
            mark[b] = true, --zero;
        bank[b].push_back(a);
    }
}
//You can add some (a, b) to the bank and make it stronger
void make_stronger(int times){
    srand(time(nullptr));
    while(times--){
        cipher_block a = random_block();
        bank[F(a)].push_back(a);
    }
}

void decode(string dir){
    ifstream cipher_text(dir);
    ofstream plain_text(dir + ".decoded");
    cipher_block a;
    int sz; //sz is the size of the vector a.v
    while(cipher_text >> sz){
        a.v.assign(sz);
        for(int i = 0; i < sz; ++i)
            cipher_text >> a.v[i];
        plain_text << (char)F(a);
    }
    cipher_text.close();
    plain_text.close();
}

void encode(string dir){
    ifstream plain_text(dir);
    ofstream cipher_text(dir + ".encoded");
    //The main reason the bank is needed:
    srand(time(nullptr));
    unsigned char c;
    while(plain_text.get(c)){
        int ind = rand() % bank[c].size();
        cipher_text << bank[c][ind].v.size() << ' ';
        for(int i = 0; i < bank[c][ind].v.size(); ++i)
            cipher_text << bank[c][ind].v[i] << ' ';
    }
    cipher_text.close();
    plain_text.close();
}
\end{lstlisting}

\myparagraph{
Also, it is possible to read the source codes of \textbf{Goorb encode 1} and \textbf{Goorb encode 2} from links in the \hyperref[subsec:codes]{\textbf{section 5.2}} for a better understanding of how this encryption style works.
}
\footnote{You can also view the codes related to this paper from the project's repository.\newline
Link of the project's repository:\newline \href{https://github.com/bistoyek-official/Goorb-encode}{https://github.com/bistoyek-official/Goorb-encode}}

\newpage

\section{Keys}

\myparagraph{
By examining the bank and the $F$ function (or some dynamic parts of $F$), it can be determined that each of them can serve as a key in this encryption process. The bank could use encryption, while $F$ could handle decryption.
}

\subsection{Pseudo-Asymmetric Property}

\myparagraph{
When considering their roles in encryption and decryption, it can be understood that they are somewhat similar to asymmetric encryption. This is because cipher-text is created by the cipher-text bank and decoded by $F$. On the other hand, cipher-text banks are obtained from $F$, and if the function is expressed as a set of ordered pairs $(a, b)$ where $a \in A$ and $b \in B$ such that $F(a) = b$, then the banks are subsets of this infinite set that meet a certain condition: each member of $B$ appears as the second member in at least one pair. The simultaneous presence of these features made me attribute the \textbf{pseudo-asymmetric} attribute to this style. Here are some possible questions that I will try to answer.
}

\subsubsection*{There are infinite banks corresponding to a particular F function, so it's not hard for hackers to guess a bank right?}

\myparagraph{
Not really. If you assume that $|B| = 256$, the chance of guessing a valid bank like $S$ even if they have $A$ and $B$ is $|B|^{-|S|}$ since we know $|B| \leq |S|$, so the greatest possible value of $|B|^{-|S|}$ is $256^{-256}$ which is almost zero. It means there is almost no chance for hackers to create a valid bank.
}

\subsubsection*{What is the point of having so many choices for banks?}

\myparagraph{
By utilizing this feature, you can assign different keys to $n$ individuals, ensuring that no two people share a common pair. This separation prevents them from discovering $F$ and decoding each other’s cipher-texts. Consequently, only you, possessing the $F$, can decode all the cipher-texts and identify the sender by reading the cipher-text. In this style, $F$ is more like a private key and banks are more like public keys. They are similar but not completely.
}

\subsubsection*{Can hackers regenerate the $F$ from valid banks?}

\myparagraph{
Since $F$ could be any function with the required conditions, it is nearly impossible for hackers to find any relation between any two pairs like $(a_1, b_1)$ and $(a_2, b_2)$, where $a_1, a_2 \in A$ and $b_1, b_2 \in B$. So even if hackers have an infinite number of valid cipher-text banks, they can't be a serious threat to the encryption's security. To keep the connection more safe, I recommend changing the banks after a while.
}

\subsubsection{Can this method be used in digital signatures?}

\myparagraph{
It can't be used in digital signatures, but it is possible to achieve a very secure encrypted communication protocol using these two keys. If two people have the same $F$, it can be used like symmetric encryption too.
}

\subsection{Raz}

\myparagraph{
It's not possible to use this encryption in digital signatures since $F$ is required to decode the cipher-texts. If we make $F$ public, the encryption loses its security. However, it is possible to achieve a secure communication protocol which I named \textbf{Raz}.
}
\footnote{Raz/raaz/: A Persian word meaning secret.}

\myparagraph{
In this protocol, two users, $X$ and $Y$, can communicate securely and send any data to each other, ensuring that only they can read the content of the messages. Initially, user $X$ has a cipher-text bank named $b_Y$ and a function $F_X$, which serves as the key for the Goorbian encryption method. Similarly, user $Y$ has a cipher-text bank named $b_X$ and a function $F_Y$. The cipher-text bank $b_X$ is generated using the function $F_Y$, and $b_Y$ is generated using $F_X$. Users can exchange new cipher-text banks or add pairs of cipher-text and plain-text to their existing banks to facilitate secure communication or even change the $F$ function and cipher-text bank at the same time.
}

\myparagraph{
Here is a simple C/C++ pseudo code to understand better how Raz works:
}

\,

\begin{lstlisting}
//In this example we are user X communicating with user Y:

const int BX = 256, BY = 256, lim = 20;
//BX and BY are number of possible values of Fx and Fy respectively
//lim is the size limit for by[i]
vector<cipher_block_y> by[BY];

//get_y(f, c) is a function that reads a cipher_block_y c from f
void add_pairs_to_by(ifstream &f){
    cipher_block_y c;
    int ind;
    while(get_y(f, c)){
        f >> ind;
        by[ind].push_back(c);
        if(lim < by[ind].size())
            by[ind].erase(by[ind].begin());
    }//If by[ind] exceeded the limit oldest element should remove
    return;
}
//write_x(f, c) is a function that writes a cipher_block_x c into f
//random_block() is a random generator that returns cipher_blocks_x
void add_pairs_to_bx(ofstream &f, int n){
    cipher_block_x c;
    srand(time(nullptr));
    while(n--){
        c = random_block();
        write_x(f, c);
        f << Fx(c) << '\n';
    }
}

void change_by(ifstream &f){
    for(int i = 0; i < BY; ++i)
        by[i].clear();
    cipher_block_y c;
    int ind;
    while(get_y(f, c)){
        f >> ind;
        if(by[ind].size() < lim)
            by[ind].push_back(c);
    }
}

void change_bx(ofstream &f){
    int cnt[BX] = {};
    cipher_block_x c;
    srand(time(nullptr));
    while(*min_element(cnt, cnt + BX) == 0){
        c = random_block();
        int res = Fx(c);
        write_x(f, c);
        f << res << '\n';
        ++cnt[res];
    }
}
//write_y(f, c) is a function that writes a cipher_block_y c into f
void encrypt_message(vector<int> pt, ofstream &f){
    srand(time(nullptr));
    //pt vector is the plain-text
    for(int i = 0; i < pt.size(); ++i)
        write_y(f, by[pt[i]][rand() % by[pt[i]].size()]);
}
//get_x(f, c) is a function that reads a cipher_block_x c from f
vector<int> decrypt_message(ifstream &f){
    vector<int> res;
    cipher_block_x c;
    while(get_x(f, c))
        res.push_back(Fx(c));
    return res;
}
\end{lstlisting}

\myparagraph{
This messaging protocol is used in a messenger named \textbf{Goorb post} and you can download and read it using links in the source codes section or checkout Goorb-encode repository in the bistoyek-official GitHub account. I wish I could explain more but due to sensitive legal aspects, I do not provide an explanation about how to work with this messenger.
}
\footnote{Link of the repository:\newline \href{https://github.com/bistoyek-official/Goorb-encode}{https://github.com/bistoyek-official/Goorb-encode}}

\myparagraph{
\textbf{Attention:}\newline
This messaging application employs some of the most advanced encryption technologies, rendering it virtually unbreakable by any conventional or quantum computing systems. Despite the robust security measures, the developer assumes no legal responsibility for any misuse or legal infringements by users or third parties. Users are solely responsible for the lawful use of this messaging application and any consequences arising therefrom.
}

\newpage

\section{Goorbian Probability Distribution}

\subsection{Explanation}

\myparagraph{
To check how long it will take for the bank to be created or become as strong as the user wants, we need to look at the issue from a probabilistic perspective. Therefore, I developed a personalized version of the \textbf{Binomial Probability Distribution} named the \textbf{Goorbian Probability Distribution}. In this distribution, there are $n$ $(1 < n)$ players doing a game, and the name of one of them is \textbf{Goorba}! At first, all of them are at coordinate $x = 0$. In every turn, a random player is chosen, and the player moves one unit to the right. The game continues for $k$ turns. This distribution's random variable is about Goorba's position on the $x$ border at the end of the $k^{th}$ turn.
}
\footnote{
For more information about Binomial Probability Distribution:\newline
\href{https://www.geeksforgeeks.org/binomial-distribution/}{https://www.geeksforgeeks.org/binomial-distribution/}
}

$$I \sim GOORB(n, k)$$

\myparagraph{
To examine how Goorba's place on the $x$ border will update, he can write $1$ if he moved one unit to the right, and write $0$ otherwise. Now suppose $s$ is a random string of his moves, where the $i^{th}$ number is $1$ with a chance of $\frac{1}{n}$ and it's $0$ with a chance of $\frac{n - 1}{n}$. So if a particular string has $O$ ones and $Z$ zeros, the chance of it happening is:
}

$$O + Z = k, \; (\frac{1}{n})^O (\frac{n - 1}{n}) ^ Z$$ 

\myparagraph{
And clearly, the $P(I = i)$ is equal to the sum of all chances of strings which have $i$ ones, so we need to know how many strings exist which satisfy this condition. It is similar to the problem of the coefficient of monomials in expressions. For a better understanding, consider the following example:
}

$$0, 1, 1, ..., 0 \; \rightarrow \; (\frac{1}{n} + \mathbf{\frac{n-1}{n}})(\mathbf{\frac{1}{n}} + \frac{n-1}{n})(\mathbf{\frac{1}{n}} + \frac{n-1}{n})...(\frac{1}{n} + \mathbf{\frac{n-1}{n}})$$

\myparagraph{
Now we know what is $f_I(i)$, but there are two other useful concepts named $e(i)$ and evacuation function $evac(i, \epsilon)$. $e(i)$ is the expected population in $x = i$ after the $k^{th}$ turn, and since all players have symmetry because they will be chosen randomly, $e(i) = n f_I(i)$. $evac(i, \epsilon)$ is the value of $k$ that satisfies $e(i) \le \epsilon$ and $k-1$ doesn't or $k = 0$ and any $k < k'$ satisfies it too. It could be shown that the answer always exists (In small values of $\epsilon$, $evac(i, \frac{\epsilon}{i+1})$ could be used as the expected value of $k$ that $x \le i$ is empty e.g. $n = 256, \, evac(0, 0.1) = 2006$).
}
\footnote{It's the same equation as what was written on page 3.}

$$P(I=i) = f_I(i) = \binom{k}{i} \frac{(n-1)^{k - i}}{n^k}, \; e(i) = n f_I(i) = \binom{k}{i} \frac{(n-1)^{k - i}}{n^{k - 1}}$$
$$evac(i, \epsilon) = k \; | \; ((\forall k' \ge k \rightarrow e(i) \le \epsilon) \; \land \; (k = 0 \lor ( k' = k - 1 \rightarrow \epsilon < e(i) ) ))$$

\myparagraph{
To better understanding, you can use the following link to check the approximate graphs of $f_I(i)$ and $e(i)$ with different values of $n$ and $k$. Since it's not easy to calculate $evac(i, \epsilon)$ (However, it can be shown that its value complexity is $O(n(i + 1)lg(\frac{n(i+1)}{min\{\epsilon,\frac{1}{2}\}})$), I created a calculator to compute it in $O(lg(n)+lg(i+1)+lg(lg(\frac{1}{min\{\epsilon,\frac{1}{2}\}})))$. The only limitation is that since numbers are bounded in computers, it can't work correctly for very large values of $n$ and $i$ and very small or large values of $\epsilon$, but theoretically, it can calculate any $evac(i, \epsilon)$. Its source code and proof of correctness can be read and downloaded from the \textbf{Goorbian Probability Distribution} directory in the project's repository.
}
\footnote{This visualization was created using the website Desmos.com to read more about it:\newline \href{https://www.desmos.com/about}{https://www.desmos.com/about}}
\footnote{In these approximated functions, a formula for calculating $ln(x!)$ called Stirling's Formula was used. To read more about it:\newline \href{https://math.mit.edu/~rmd/440/stirlingformula.pdf}{https://math.mit.edu/$\sim$rmd/440/stirlingformula.pdf}}

\myparagraph{
Link of the visualized diagrams:\newline \; \href{https://www.desmos.com/calculator/oh6mmliiik}{https://www.desmos.com/calculator/oh6mmliiik}
}

\myparagraph{
Link of the project's repository:\newline \;
\href{https://github.com/bistoyek-official/Goorb-encode}{https://github.com/bistoyek-official/Goorb-encode}
}

\subsection{Its Applications}

\myparagraph{
Now suppose the number of players is $|B|$, and when a random and necessarily new $a$ is chosen and $F(a) = b$, the size of $bank_b$ will increase by one, and we can move the corresponding player to $b$ one unit to the right. Since creating $a$ is random, there is no difference with the past version of the game from a probabilistic view, and it's even possible to create a bank by parallel processing it with $p$ processors. It would be created $p$ times faster than the single processor case. And now we know the answer to the first question asked in this section.
}

\myparagraph{
\textbf{Note:} In this entire process, it is assumed that each time a different random value is produced from the generated values, given the huge size of the set of possible cipher-blocks and the \textbf{Birthday problem} that shows it's needed to use the random generator $O(\sqrt{N})$ times to have a good chance of collision. With a chance approximately equal to one, a collision never happens. ($N$ is the number of possible cipher-blocks that the $random\_block$ function can make.)
}
\footnote{
For more information about the Birthday problem:\newline
\href{https://www.oxfordreference.com/display/10.1093/oi/authority.20110803095508254}{https://www.oxfordreference.com/display/10.1093/oi/authority.20110803095508254}
}

\myparagraph{
Also, if the path traveled by each person can affect the probability of being chosen, this distribution can be used in many problems.
}

\newpage

\section{In The Face Of Attacks}

\myparagraph{
In this section, I'll examine the resilience and robustness of the Goorbian encryption method against various types of attacks. My focus is on two distinct versions of this method: \textbf{Goorb encode 1} and \textbf{Goorb encode 2}. The reason for examining two versions is that, in this method, the key is not a simple number but some parts of an irreversible function, named the $F$ function. Since the only necessary condition for the $F$ function is that it achieves a uniform distribution when mapping an infinite set to a finite set, analyzing and attacking an unknown, complex function such as this presents unique challenges. Additionally, the structure of $F$ and the cipher blocks can vary in complexity, from something as simple as $F(x) = x \, mod \, 2$, where $x$ is an integer, to more intricate and hard-to-guess functions.
}

\myparagraph{
It can be shown that there are some possible cases of $F$ which crypt analysis and statistical analysis attacks are totally ineffective and even quantum attacks are ineffective so it's nearly impossible for attackers to decrypt a random cipher-text with desired accuracy, e.g. AES is resistant to quantum attacks and it's possible to define a series of functions that $F_i : \{0, 1\}^{128} \rightarrow \{0, 1\} $ that $F_i(x) = [i^{th} \text{ bit of } AES_{128 \rightarrow 128}(x \, mod \, 2^{128})] \; (1 \le i \le 128)$. Then there is at least one of these functions immune to these attacks because if it was possible to guess all the functions' outputs with desired accuracy, then due to the production principle, if someone performs these attacks with the goal of achieving accuracy greater or equal to $0.9999$, attackers can break the AES because they can merge $F_i(x) \, (1 \le i \le 128)$ and guess the plain-text with accuracy equal to $0.9999^{128} \approx 0.987$. Given that AES cryptography is resistant to quantum attacks, it's impossible. Therefore, I decided to subject these two versions to various tests. In two scenarios, in one it's known the structure of the function $F$ besides the key (Let call it $F-$key), while in the other scenario, there is no clue about the encryption at all.
}
\footnote{For more information about crypt analysis attacks:\newline \href{https://www.geeksforgeeks.org/cryptanalysis-and-types-of-attacks/}{https://www.geeksforgeeks.org/cryptanalysis-and-types-of-attacks/}}
\footnote{For more information about statistical analysis attacks:\newline \href{https://link.springer.com/chapter/10.1007/978-3-030-95161-0_5}{https://link.springer.com/chapter/10.1007/978-3-030-95161-0\_5}}
\footnote{For more information about quantum attacks:\newline \href{https://coingape.com/everything-you-need-to-know-about-quantum-attacks/}{https://coingape.com/everything-you-need-to-know-about-quantum-attacks/}}
\footnote{For more information about AES:\newline \href{https://www.geeksforgeeks.org/advanced-encryption-standard-aes/}{https://www.geeksforgeeks.org/advanced-encryption-standard-aes/}}
\footnote{For more information about post quantum (anti quantum attacks) cryptography: \newline \href{https://www.nist.gov/cybersecurity/what-post-quantum-cryptography}{https://www.nist.gov/cybersecurity/what-post-quantum-cryptography}}

\subsection{No clue scenario}

\myparagraph{
Imagine there is no clue about cryptography but it is guaranteed that the key(s) of it won't change in all of the attack time (clearly now it's an easier problem), and there is an oracle that could answer these queries:\newline \newline$1$. For $a$ such $a \in A$ what is $F(a)$. (costs $x$)\newline$2$. For $b$ that $b \in B$ return a random $a \in A$ which $F(a) = b$. (costs $y$)
}

\myparagraph{
Using these two queries, it is possible to perform any crypt-analysis and statistical-analysis attacks, and since we have no data about the $F$ function, a combined attack with one of the crypt-analysis attacks and ML attacks (ML attacks are in the statistical attacks category) could be a nice try. So, I designed some of these attacks and attacked these two versions to test their security with each key. The source codes of attacks are available in the project's repository in the \textbf{Goorb Lab} directory, and here are the results of the tests.
}
\footnote{\textbf{Goorb Lab}'s source code is available in the \textbf{Goorb Lab} directory in the project's repository. This attack simulator can help people to test their encryption algorithms or their attack algorithms much easier. \newline Link of the Goorb Lab's source code: \newline \href{https://github.com/bistoyek-official/Goorb-encode/tree/main/Goorb\%20Lab}{https://github.com/bistoyek-official/Goorb-encode/tree/main/Goorb\%20Lab}\newline \textbf{Disclaimer:} This project developed by \textbf{bistoyek R.I.C.} is intended solely for educational and research purposes. It should not be used for any malicious or illegal activities. \textbf{Goorb Lab} and its author are not responsible for any misuse of the information provided or any consequences resulting from its application.}

\myparagraph{
Using the \textbf{Goorb Lab}, my simple cryptography lab, nine different combined attacks were conducted on the \textbf{Goorb encode 1} and \textbf{Goorb encode 2} algorithms, which were set with \textbf{key-1} and \textbf{key-0}, respectively. Each attack was repeated under various conditions and states to evaluate the actual performance of these algorithms against diverse threats. Below, using the provided links, the performance of algorithms against these attacks can be observed (For readers using the printed version who can't access the hyperlinks, the results can be accessed at: \href{https://github.com/bistoyek-official/Goorb-encode/tree/main/Goorb\%20Lab/Encryption/Performance}{https://github.com/bistoyek-official/Goorb-encode/tree/main/Goor \newline b\%20Lab/Encryption/Performance}):
}

\subsubsection*{Goorb encode 1 performance:}

\begin{center}
\begin{tabular}{|>{\centering\arraybackslash}m{3cm}|>{\centering\arraybackslash}m{3cm}|>{\centering\arraybackslash}m{3cm}|} 
\hline
\rowcolor{cyan!20} \textbf{MLP-Linear} & \textbf{MLP-Differential} & \textbf{MLP-Known-plain-text} \\
\hline
\rowcolor{white} \href{https://github.com/bistoyek-official/Goorb-encode/tree/main/Goorb\%20Lab/Encryption/Performance/Goorb\%20encode\%201/MLP/Linear}{results} & \href{https://github.com/bistoyek-official/Goorb-encode/tree/main/Goorb\%20Lab/Encryption/Performance/Goorb\%20encode\%201/MLP/Differential}{results} & \href{https://github.com/bistoyek-official/Goorb-encode/tree/main/Goorb\%20Lab/Encryption/Performance/Goorb\%20encode\%201/MLP/Known-plain-text}{results} \\
\hline
\rowcolor{cyan!20} \textbf{CNN-Linear} & \textbf{CNN-Differential} & \textbf{CNN-Known-plain-text} \\
\hline
\rowcolor{white} \href{https://github.com/bistoyek-official/Goorb-encode/tree/main/Goorb\%20Lab/Encryption/Performance/Goorb\%20encode\%201/CNN/Linear}{results} & \href{https://github.com/bistoyek-official/Goorb-encode/tree/main/Goorb\%20Lab/Encryption/Performance/Goorb\%20encode\%201/CNN/Differential}{results} & \href{https://github.com/bistoyek-official/Goorb-encode/tree/main/Goorb\%20Lab/Encryption/Performance/Goorb\%20encode\%201/CNN/Known-plain-text}{results} \\
\hline
\rowcolor{cyan!20} \textbf{RNN-Linear} & \textbf{RNN-Differential} & \textbf{RNN-Known-plain-text} \\
\hline
\rowcolor{white} \href{https://github.com/bistoyek-official/Goorb-encode/tree/main/Goorb\%20Lab/Encryption/Performance/Goorb\%20encode\%201/RNN/Linear}{results} & \href{https://github.com/bistoyek-official/Goorb-encode/tree/main/Goorb\%20Lab/Encryption/Performance/Goorb\%20encode\%201/RNN/Differential}{results} & \href{https://github.com/bistoyek-official/Goorb-encode/tree/main/Goorb\%20Lab/Encryption/Performance/Goorb\%20encode\%201/RNN/Known-plain-text}{results} \\
\hline 
\end{tabular} 
\end{center}

\subsubsection*{Goorb encode 2 performance:}

\begin{center}
\begin{tabular}{|>{\centering\arraybackslash}m{3cm}|>{\centering\arraybackslash}m{3cm}|>{\centering\arraybackslash}m{3cm}|} 
\hline
\rowcolor{cyan!20} \textbf{MLP-Linear} & \textbf{MLP-Differential} & \textbf{MLP-Known-plain-text} \\
\hline
\rowcolor{white} \href{https://github.com/bistoyek-official/Goorb-encode/tree/main/Goorb\%20Lab/Encryption/Performance/Goorb\%20encode\%202/MLP/Linear}{results} & \href{https://github.com/bistoyek-official/Goorb-encode/tree/main/Goorb\%20Lab/Encryption/Performance/Goorb\%20encode\%202/MLP/Differential}{results} & \href{https://github.com/bistoyek-official/Goorb-encode/tree/main/Goorb\%20Lab/Encryption/Performance/Goorb\%20encode\%202/MLP/Known-plain-text}{results} \\
\hline
\rowcolor{cyan!20} \textbf{CNN-Linear} & \textbf{CNN-Differential} & \textbf{CNN-Known-plain-text} \\
\hline
\rowcolor{white} \href{https://github.com/bistoyek-official/Goorb-encode/tree/main/Goorb\%20Lab/Encryption/Performance/Goorb\%20encode\%202/CNN/Linear}{results} & \href{https://github.com/bistoyek-official/Goorb-encode/tree/main/Goorb\%20Lab/Encryption/Performance/Goorb\%20encode\%202/CNN/Differential}{results} & \href{https://github.com/bistoyek-official/Goorb-encode/tree/main/Goorb\%20Lab/Encryption/Performance/Goorb\%20encode\%202/CNN/Known-plain-text}{results} \\
\hline
\rowcolor{cyan!20} \textbf{RNN-Linear} & \textbf{RNN-Differential} & \textbf{RNN-Known-plain-text} \\
\hline
\rowcolor{white} \href{https://github.com/bistoyek-official/Goorb-encode/tree/main/Goorb\%20Lab/Encryption/Performance/Goorb\%20encode\%202/RNN/Linear}{results} & \href{https://github.com/bistoyek-official/Goorb-encode/tree/main/Goorb\%20Lab/Encryption/Performance/Goorb\%20encode\%202/RNN/Differential}{results} & \href{https://github.com/bistoyek-official/Goorb-encode/tree/main/Goorb\%20Lab/Encryption/Performance/Goorb\%20encode\%202/RNN/Known-plain-text}{results} \\
\hline 
\end{tabular} 
\end{center}

\newpage

\subsection{Known $F-$key scenario}

\myparagraph{
In the previous section, it was demonstrated that even with a fixed key, there exist algorithms based on this method that exhibit strong resistance against relatively powerful attacks. This indicates that such methods maintain high security even with a fixed key. It will now be shown that there exist keys that can transform these algorithms into post-quantum algorithms, even if the underlying function $F-$key is not post-quantum secure.
}

\myparagraph{
For example, in cases where the entire dynamic part of $F$ (the key), is at the end of the structure of $F$, and $F-$key is a function from $A$ to $S = \{0, 1\}^k$ which $256 \le k$ and $2^k \, mod \, |B| = 0$, a key could be adjusted such that the encryption becomes post-quantum secure, given that $F-$key uniformly distributes $A$ over the set $S$.
}

\myparagraph{
\textbf{One of the ways to construct a key that makes the encryption post-quantum is to use the following unsolved problem:}
}

\myparagraph{
In the context of functions $g: \{0, 1\}^{256} \rightarrow \{0, 1\}$ where the output is the XOR of bits from a secure hash function based on the SHA-256 algorithm applied to the input, it is known that no classical or quantum algorithm currently has been discovered that can determine the exact form of such functions merely from their input-output pairs, due to their inherent complexity and the principles of cryptographic hashes security.
}

\myparagraph{
Considering the infinite number of such functions $g$ that exist (Because there is infinite number of cryptographic secure hashes based on SHA-256), if $|B| = 2 ^ l$ ($l \le 256$), it's possible to utilize this unsolved problem to construct keys that meet the condition. Specifically, it's possible to choose $l$ functions like $g$ and set the key like below:
}

$$F(a) = \textnormal{key}(x) = \sum_{i=0}^{l-1} g_i(x\, mod \, 2^{256}) \cdot 2 ^ i $$
$$ \text{where} \; (F - \text{key})(a) = x \; \textnormal{and} \; x \in S$$

\myparagraph{
\textbf{Note:} To obtain a desirable result, for any two $g_i(x)$ and $g_j(x)$ $(0 \le i  < j < l)$, the probability that the output is $1$ for a random $x$ should be independent events, and for each function in this functional series, the probability that the output equals $1$ should be $\frac{1}{2}$, both with a negligible error range (For more explanation, check out the \textbf{Goorbian Probability Distribution/Key Analysis} directory in the project's repository.). For small values of $|B| = 2^l$, the creation of such series is not so challenging, e.g., for $l \le 8$, one of the ways is to set $g_i(x)$ equal to XOR of all bits of SHA-256$^{(i+1)}(x) \And x$.
}

\myparagraph{
So even if the structure of $F-$key is known to a hacker, keys can still be created for a set of $F-$key structures that make the encryption post-quantum secure. Therefore, even with the knowledge of the $F-$key structure, the security of the encryption cannot be threatened by a hacker.
}

\newpage

\section{Conclusion}

\subsection{Goorbian encryption's Applications}

\myparagraph{
In the \textbf{Goorb encode} project, I introduced and evaluated a novel encryption method called \textbf{Goorbian encryption}. One of the key aspects of this method is the security of the encryption banks and the ability to approximate the cost of constructing cipher-texts banks using \textbf{Goorbian Probability Distribution}. This analysis helped us better understand the computational cost required to achieve the desired state for a particular encryption scheme.
}

\myparagraph{
The \textbf{pseudo-asymmetric} property of keys in \textbf{Goorbian encryption} allows us to achieve the \textbf{Raz} encryption protocol. This protocol significantly enhanced the security of communications. The developed messaging application, \textbf{Goorb post}, provided a secure and dynamic environment for data exchange where users could easily change their encryption keys and algorithms.
}

\myparagraph{
In the simple cryptography laboratory, \textbf{Goorb Lab}, I specifically examined the security of two encryption algorithms, \textbf{Goorb encode 1} and \textbf{Goorb encode 2}, that utilized this new style of encryption. These tests demonstrated that the \textbf{Goorbian encryption} method can provide significant security.
}

\myparagraph{
The applications of \textbf{Goorbian encryption} highlight its high potential and can serve as a foundation for future studies and developments in the field of information security and cryptography. This project not only presented a new and effective method in cryptography but also provided practical protocols and tools for real-world applications. It was also shown that many $F$ functions in encryption can be equipped with post-quantum keys.
}

\subsubsection*{Suggestions for Future Researches:}

\begin{enumerate}
\item \textbf{Development and Optimization of Algorithms:}
\begin{itemize}
\item Develop and improve the performance and efficiency of encryption algorithms based on Goorbian encryption.
\item Create and optimize powerful and efficient $F$ functions and keys.
\end{itemize}
\item \textbf{Industrial and Commercial Applications:}
\begin{itemize} 
\item Assess practical applications of Goorbian encryption in various industries such as cybersecurity, cryptocurrencies, and telecommunications. 
\item Study real-world use cases and implement this method in existing systems.
\end{itemize}
\item \textbf{Examination and Analysis of Emerging Threats:}
\begin{itemize}
\item Study and evaluate emerging threats and propose solutions to counter them.
\item Analyze the impact and potential outcomes of new threats on the security of Goorbian encryption.
\end{itemize}
\end{enumerate}

\newpage

\label{subsec:codes}
\subsection{Source Codes}

\myparagraph{
The following section provides links to the source codes related to the project. The repository contain the implementation details of the \textbf{Goorbian encryption} methods and associated tools. You can explore and analyze the code to gain a deeper understanding of the algorithms and their functionalities:
}

\begin{itemize}
\item \textbf{Goorb encode 1:} \\ \href{https://github.com/bistoyek-official/Goorb-encode/tree/main/Goorb\%20enco \newline de\%201}{https://github.com/bistoyek-official/Goorb-encode/tree/main/Goorb\%20 \newline encode\%201}
\item \textbf{Goorb encode 2:} \\ \href{https://github.com/bistoyek-official/Goorb-encode/tree/main/Goorb\%20encode\%202}{https://github.com/bistoyek-official/Goorb-encode/tree/main/Goorb\%20 \newline encode\%202} 
\item \textbf{Goorb post:} \\ \href{https://github.com/bistoyek-official/Goorb-encode/tree/main/Goorb\%20post}{https://github.com/bistoyek-official/Goorb-encode/tree/main/Goorb\%20 \newline post} 
\item \textbf{Goorb Lab:} \\ \href{https://github.com/bistoyek-official/Goorb-encode/tree/main/Goorb\%20Lab}{https://github.com/bistoyek-official/Goorb-encode/tree/main/Goorb\%20 \newline Lab}
\end{itemize}

\myparagraph{
Link of the \textbf{Goorb encode} project's repository:\\
\href{https://github.com/bistoyek-official/Goorb-encode}{https://github.com/bistoyek-official/Goorb-encode}
}

\newpage

\section{References}

\myparagraph{
[1] \href{https://ssldragon.com/blog/public-key-cryptography}{https://ssldragon.com/blog/public-key-cryptography}
}

\myparagraph{
[2] \href{https://networkencyclopedia.com/sha-256-unmasked-deciphering-cryptographic-hash-functions}{https://networkencyclopedia.com/sha-256-unmasked-deciphering-cryptogra \newline phic-hash-functions}
}

\myparagraph{
[3] \href{https://games.skillz.com/guides/bubble-shooter}{https://games.skillz.com/guides/bubble-shooter}
}

\myparagraph{
[4] \href{https://github.com/bistoyek-official/Goorb-encode}{https://github.com/bistoyek-official/Goorb-encode}
}

\myparagraph{
[5] \href{https://www.geeksforgeeks.org/binomial-distribution/}{https://www.geeksforgeeks.org/binomial-distribution/}
}

\myparagraph{
[6] \href{https://www.desmos.com/about}{https://www.desmos.com/about}
}

\myparagraph{
[7] \href{https://math.mit.edu/~rmd/440/stirlingformula.pdf}{https://math.mit.edu/$\sim$rmd/440/stirlingformula.pdf}
}

\myparagraph{
[8] \href{https://www.oxfordreference.com/display/10.1093/oi/authority.20110803095508254}{https://www.oxfordreference.com/display/10.1093/oi/authority.2011080309\newline 254}
}

\myparagraph{
[9] \href{https://www.geeksforgeeks.org/cryptanalysis-and-types-of-attacks/}{https://www.geeksforgeeks.org/cryptanalysis-and-types-of-attacks/}
}

\myparagraph{
[10] \href{https://link.springer.com/chapter/10.1007/978-3-030-95161-0_5}{https://link.springer.com/chapter/10.1007/978-3-030-95161-0\_5}
}

\myparagraph{
[11] \href{https://coingape.com/everything-you-need-to-know-about-quantum-attacks/}{https://coingape.com/everything-you-need-to-know-about-quantum-attack\newline s/}
}

\myparagraph{
[12] \href{https://www.geeksforgeeks.org/advanced-encryption-standard-aes/}{https://www.geeksforgeeks.org/advanced-encryption-standard-aes/}
}

\myparagraph{
[13] \href{https://www.nist.gov/cybersecurity/what-post-quantum-cryptography}{https://www.nist.gov/cybersecurity/what-post-quantum-cryptography}
}

\end{document}