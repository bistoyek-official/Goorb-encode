\documentclass[oneside]{book}
\usepackage[utf8]{inputenc}
\usepackage{hyperref}
\usepackage{mathtools}
\usepackage{listings}
\usepackage{xcolor}  
\usepackage{graphicx}
\usepackage{titlesec}
\usepackage{cleveref}
\usepackage{fancyhdr}
\usepackage{tcolorbox}

\pagestyle{fancy}
\fancyhf{}
\fancyhead[L]{\textbf{Key security}}
\fancyhead[R]{\thepage}
\renewcommand{\headrulewidth}{0pt}

\hypersetup{
    colorlinks=true,
    linkcolor=black,
    filecolor=magenta,
    urlcolor=cyan,
    citecolor=green
}

\titleformat{\chapter}[hang]{\Huge\bfseries}{}{0pt}{}
\titleformat{\section}[hang]{\Large\bfseries}{}{0pt}{\thesection \space}

\renewcommand{\thesection}{\arabic{section}}

\lstset{
    language=C++,
    backgroundcolor=\color{black!5},
    basicstyle=\ttfamily\footnotesize,
    keywordstyle=\color{blue}\bfseries,
    commentstyle=\color{green},
    stringstyle=\color{red},
    numbers=left,
    numberstyle=\tiny\color{gray},
    stepnumber=1,
    numbersep=10pt,
    showspaces=false,
    showstringspaces=false,
    showtabs=false,
    frame=single,
    tabsize=4,
    captionpos=b,
    breaklines=true,
    breakatwhitespace=false,
    title=\lstname
}

\newcommand{\myparagraph}[1]{\paragraph{\textnormal{#1}}}

\begin{document}

\small

\section*{Argument}

\myparagraph{
Assuming $|B|=2^l$ and there are $l$ functions $g_i(x): \{0, 1\}^{256} \rightarrow \{0, 1\} \; (0 \le i < l)$. If the probability that the output is $1$ for a random $x$ should be independent events, and for each function in this functional series, the probability that the output equals $1$ is $\frac{1}{2}$, then $F$ distributes $A$ uniformly on $B$. In fact, these two statements are equivalent. This is important because it's known that each of these functions is undetectable individually, and we don't want to make them easier to detect when the hacker can query all $l$ functions. When $F$ has a uniform distribution, it's impossible for hackers to achieve a function like $h$ where knowing some of the $g_i(x)$s can't guess an unknown $g_j(x)$ with an accuracy grater than $\frac{1}{2}$.
}

\myparagraph{
If in the second query, the hacker can find out $F(a)$ where $a \in$ cipher-text bank, we can use a bank with equal $X_i$s (I suggest doing it for more security against known cipher-text attacks). $X_i$ is the size of $bank[i], \; i \in B$. Otherwise, it's necessary to achieve a good key that distributes $A$ more uniformly. With numerical tests on $g_i$ functions, it is possible to understand how far from ideal conditions we are, but it doesn't provide a certain quantity, so I offer these two tests.
}

\section{Tail Test}

\myparagraph{
In this test, the expected population of cell number $0$ is examined in the ideal case (shown as $e(0)$) and for a particular key (shown as $e'(0)$). This utilizes the fact that with $1 < k$, an array $<a_i>$ of positive numbers with a fixed value of $\sum a_i$ has the minimum value of $\sum {a_i}^k$ when all $a_i$s are equal.
}

$$I \sim GOORB(n, k)$$
$$e(0) = \frac{(n-1)^k}{n^{k-1}} = n \frac{(n-1)^k}{n^k} = n P^k, \; P = \frac{n-1}{n}$$

\myparagraph{
For a particular key, $e'(0)$ can be computed by summing the probability that $b$ won't be chosen in any round after the $k^{th}$ round, for all $b \in B$. It can be computed by experimental methods and can be displayed as $nQ^k, \; (0 < Q < 1)$, and since in each round exactly one element will be chosen, $\sum (1 - p_i) = n - 1$ holds for both cases ($p_i$ is the probability that the $i^{th}$ element will be chosen in a round).
}

$$e'(0) = \sum_{i=1}^{n} (1 - p_i)^k = n Q^k$$

\myparagraph{
Now, it's known that $e(0) < e'(0)$, so a parameter named safety\_rate is defined as below. When the safety rate is closer to $1$, it means we achieved a safer key; otherwise:
}

$$nP^k \le nQ^k \rightarrow P^k \le Q^k$$
$$\textnormal{safety\_rate} \in (0, 1], \; \textnormal{safety\_rate} = 1 - Q^k + P^k$$

\section{Head Test}

\myparagraph{
In this test, a quantity $W$ is designed to measure What is the density of players in cells nearby $x=k$ in the ideal case and for a particular key and comparing them together. It's value for the ideal case is calculated below, and using experimental methods it's possible to approximate it for that particular key.
}

$$E[W] = M, \; W = \sum_{i=1}^n n^{X_i}$$
$$X_i = \textnormal{number of times } i^{th} \textnormal{ player has been chosen}$$

\myparagraph{
Then I prove $M \le M'$ ($M'$ is the value of $E[W]$ for a particular key). Considering the simplification of the following equation, the minimum value of $E[W]$ is achieved when the distribution is uniform, and the maximum achieved when their variance is maximized. Therefore, $min\{M'\}=M$, or $M \le M'$.
}

$$E[W] = \sum_{i=1}^n \sum_{j=0}^k n^j {p_i}^j (1-p_i)^{k-j} \binom{k}{j} = \sum_{i=1}^n \left( (1-p_i)^k \sum_{j=0}^k \left( \frac{np_i}{1-p_i} \right)^j \binom{k}{j} \right)$$

$$= \sum_{i=1}^n (1-p_i)^k \left( \frac{np_i}{1-p_i} + 1 \right)^k = \sum_{i=1}^n ((n-1)p_i + 1)^k$$

\myparagraph{
 And again, same as the previous test, a parameter is defined to measure how safe the key is:
}

$$\textnormal{safety\_rate} \in (0, 1], \; \textnormal{safety\_rate} = 1 - \frac{M' - M}{n^k}$$

\end{document}